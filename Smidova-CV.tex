%%%%%%%%%%%%%%%%%%%%%%%%%%%%%%%%%%%%%%%%%
% Compact Academic CV
% LaTeX Template
% Version 2.0 (6/7/2019)
%
% This template originates from:
% https://www.LaTeXTemplates.com
%
% Authors:
% Dario Taraborelli (http://nitens.org/taraborelli/home)
% Vel (vel@LaTeXTemplates.com)
%
% License:
% CC BY-NC-SA 3.0 (http://creativecommons.org/licenses/by-nc-sa/3.0/)
%
%%%%%%%%%%%%%%%%%%%%%%%%%%%%%%%%%%%%%%%%%

%----------------------------------------------------------------------------------------
%	PACKAGES AND OTHER DOCUMENT CONFIGURATIONS
%----------------------------------------------------------------------------------------

\documentclass[11pt]{article} % Default document font size

%%%%%%%%%%%%%%%%%%%%%%%%%%%%%%%%%%%%%%%%%
% Compact Academic CV
% Structural Definitions
% Version 1.0 (6/7/2019)
%
% This template originates from:
% https://www.LaTeXTemplates.com
%
% Authors:
% Dario Taraborelli (http://nitens.org/taraborelli/home)
% Vel (vel@LaTeXTemplates.com)
%
% License:
% CC BY-NC-SA 3.0 (http://creativecommons.org/licenses/by-nc-sa/3.0/)
%
%%%%%%%%%%%%%%%%%%%%%%%%%%%%%%%%%%%%%%%%%

%----------------------------------------------------------------------------------------
%	REQUIRED PACKAGES AND MISC CONFIGURATIONS
%----------------------------------------------------------------------------------------

\usepackage{graphicx} % Required for including images

\setlength{\parindent}{0pt} % Stop paragraph indentation

%----------------------------------------------------------------------------------------
%	MARGINS
%----------------------------------------------------------------------------------------

\usepackage{geometry} % Required for adjusting page dimensions and margins

\geometry{
	paper=a4paper, % Paper size, change to letterpaper for US letter size
	top=3.25cm, % Top margin
	bottom=4cm, % Bottom margin
	left=3.5cm, % Left margin
	right=3.5cm, % Right margin
	headheight=0.75cm, % Header height
	footskip=1cm, % Space from the bottom margin to the baseline of the footer
	headsep=0.75cm, % Space from the top margin to the baseline of the header
	%showframe, % Uncomment to show how the type block is set on the page
}

%----------------------------------------------------------------------------------------
%	FONTS
%----------------------------------------------------------------------------------------

\usepackage[utf8]{inputenc} % Required for inputting international characters
\usepackage[T1]{fontenc} % Output font encoding for international characters

\usepackage[semibold]{ebgaramond} % Use the EB Garamond font with a reduced bold weight

%----------------------------------------------------------------------------------------
%	SECTION STYLING
%----------------------------------------------------------------------------------------

\usepackage{sectsty} % Allows changing the font options for sections in a document

\sectionfont{\fontsize{13.5pt}{18pt}\selectfont} % Set font options for sections
\subsectionfont{\mdseries\scshape\normalsize} % Set font options for subsections
\subsubsectionfont{\mdseries\upshape\bfseries\normalsize} % Set font options for subsubsections
\usepackage{multicol}
%----------------------------------------------------------------------------------------
%	MARGIN YEARS
%----------------------------------------------------------------------------------------

\usepackage{marginnote} % Required to output text in the margin

\newcommand{\years}[1]{\marginnote{\scriptsize #1}} % New command for adding years to the margin
\renewcommand*{\raggedleftmarginnote}{} % Left-align the years in the margin
\setlength{\marginparsep}{-10pt} % Move the margin content closer to the text
\reversemarginpar % Margin text to be output into the left margin instead of the default right margin

%----------------------------------------------------------------------------------------
%	COLOURS
%----------------------------------------------------------------------------------------

\usepackage[usenames, dvipsnames]{xcolor} % Required for specifying colours by name

%----------------------------------------------------------------------------------------
%	LINKS
%----------------------------------------------------------------------------------------

\usepackage[bookmarks, colorlinks, breaklinks]{hyperref} % Required for links

% Set link colours
\hypersetup{
	linkcolor=blue,
	citecolor=blue,
	filecolor=black,
	urlcolor=MidnightBlue
}

\newcounter{barcount}

% Environment to hold a new bar chart
\newenvironment{barchart}[1]{ % The only parameter is the maximum bar width, in cm
	\newcommand{\barwidth}{0.35}
	\newcommand{\barsep}{0.2}
	
	% Command to add a bar to the bar chart
	\newcommand{\baritem}[2]{ % The first argument is the bar label and the second is the percentage the current bar should take up of the total width
		\pgfmathparse{##2}
		\let\perc\pgfmathresult
		
		\pgfmathparse{#1}
		\let\barsize\pgfmathresult
		
		\pgfmathparse{\barsize*##2/100}
		\let\barone\pgfmathresult
		
		\pgfmathparse{(\barwidth*\thebarcount)+(\barsep*\thebarcount)}
		\let\barx\pgfmathresult
		
		\filldraw[fill=black, draw=none] (0,-\barx) rectangle (\barone,-\barx-\barwidth);
		
		\node [label=180:{\textcolor{black}{##1}}] at (0,-\barx-0.175) {};
		\addtocounter{barcount}{1}
	}
	\begin{tikzpicture}
	\setcounter{barcount}{0}
}{
	\end{tikzpicture}
}

\usepackage{tikz} % Required for creating the plots
\usetikzlibrary{shapes, backgrounds}
\tikzset{x=1cm, y=1cm} % Default tikz units

% Command to vertically centre adjacent content
\newcommand{\vcenteredhbox}[1]{% The only parameter is for the content to centre
	\begingroup%
	\setbox0=\hbox{#1}\parbox{\wd0}{\box0}%
	\endgroup%
}

 % Include the file specifying the document structure and styling

% Set PDF meta-information
\hypersetup{
	pdftitle={Lucia Smidova - Curriculum vitae},
	pdfauthor={Lucia Smidova}
}

%----------------------------------------------------------------------------------------

\begin{document}

%----------------------------------------------------------------------------------------
%	CONTACT AND GENERAL INFORMATION
%----------------------------------------------------------------------------------------

{\LARGE\bfseries Lucia Šmídová} \\ % Name

Born: November 12, 1993---Trenčín, Slovakia\\ % Date of birth
Nationality: Slovak% Nationality

\begin{multicols}{2}
	\section*{Contact information}
	Institute of Geology and Palaeontology\\ % Address
	Faculty of Natural Sciences, Charles University\\ Albertov 6, 18200 Czech republic
	\medskip % Whitespace
	




Phone: +420 727 885 153\ % Phone number


Email: \href{mailto:smidovaluc@natur.cuni.cz}{smidovaluc@natur.cuni.cz}\\ % Email address
Personal website: \href{https://luci-netizen.github.io/personal_website/index.html}{https://luci-netizen.github.io/}\\ % Academic/personal website

\vspace{0.03\textheight} % Whitespace between contact information and specific CV information

%------------------------------------------------


%------------------------------------------------

\section*{Current position}

\emph{PhD student}, Institute of Geology and Palaeontology, Charles University % Current or most recent employment position

%------------------------------------------------

\section*{Areas of specialisation}

Paleoentomology, Fossil cockroaches, Evolution of Insects % Primary areas of research interest

\end{multicols}
%----------------------------------------------------------------------------------------
%	EDUCATION
%----------------------------------------------------------------------------------------

\section*{Education}

\years{2018}\textsc{MSc} in Geology, Charles University, Prague\\
\years{2018- now}\textsc{PhD} in Geology, Charles University, Prague


%----------------------------------------------------------------------------------------
%	PUBLICATIONS AND TALKS
%----------------------------------------------------------------------------------------

\section*{Publications \& conferences}

\subsection*{Journal articles}

\years{2016}Vršanský, P.V., \textbf{Šmídová, L}., Valaška, D., Barna, P., Vidlička, Ľ., Takáč, P., Pavlik, L., Kúdelová, T., Karim, T.S., Zelagin, D. and Smith, D., 2016. Origin of origami cockroach reveals long-lasting (11 Ma) phenotype instability following viviparity. The Science of Nature, 103(9), pp.1-15. \\
\years{2017}\textbf{Šmídová, L}. and Lei, X., 2017. The earliest amber-recorded type cockroach family was aposematic (Blattaria: Blattidae). Cretaceous Research, 72, pp.189-199.\\
\years{2018}Vršanský, P., Bechly, G., Zhang, Q., Jarzembowski, E.A., Mlynský, T., \textbf{Šmídová, L.}, Barna, P., Kúdela, M., Aristov, D., Bigalk, S. and Krogmann, L., 2018. Batesian insect-insect mimicry-related explosive radiation of ancient alienopterid cockroaches. Biologia, 73(10), pp.987-1006.\\
\years{2019}Vršanský, P., \textbf{Šmídová, L}., Sendi, H., Barna, P., Müller, P., Ellenberger, S., Wu, H., Ren, X., Lei, X., Azar, D. and Šurka, J., 2019. Parasitic cockroaches indicate complex states of earliest proved ants. Biologia, 74(1), pp.65-89.\\
\years{2019}Vršanský, P., Vršanská, L., Beňo, M., Bao, T., Lei, X.J., Ren, X.J., Wu, H., \textbf{Šmídová, L}., Bechly, G., Jun, L. and Yeo, M., 2018. Pathogenic DWV infection symptoms in a Cretaceous cockroach. Paleontographica Abteilung A, pp.0375-0442.\\
\years{2019}Barna, P., \textbf{Šmídová, L}. and José, M.A.C., 2019. Living cockroach genus Anaplecta discovered in Chiapas amber (Blattaria: Ectobiidae: Anaplecta vega sp. n.). PeerJ, 7, p.e7922.\\
\years{2020} \textbf{Šmídová, L}., 2020. Cryptic bark cockroach (Blattinae: Bubosa poinari gen. et sp. nov.) from mid-Cretaceous amber of northern Myanmar. Cretaceous Research, 109, p.104383.\\
\years{2021}\textbf{Šmídová L}., 2021. New genus and species of the families Olidae and Corydiidae (Corydioidea, Blattodea) from mid-Cretaceous Kachin amber. Palaentographica Abteilung A (in press)\\
\years{2021} \textbf{Šmídová L.}, Vidlička, Ľ., Wiedmann, S. 2021. Appearance of the family Blaberidae (Insecta: Blattaria) during the Cretaceous and a review of fossils of this family. Palaentographica Abteilung A (in press)\begin{flushleft}
\end{flushleft}

\subsection*{Conferences}

\years{2017} International Conference Of Young Geologists, Herlany-Dobczyce (Poland), March 30th – April 2nd 2017.\\
\years{2020} XXIst International Conference of Young Geologists 2020. 5. - 6., November 2020, Online:
Šmídová L. 2020. Cockroach fauna of mid-Cretaceous Myanmar amber as valuable insight into the phylogeny of the group [abstract].



%------------------------------------------------

\subsection*{Popularization}

\years{2016} The pop-science article about newly described species in 2015, \href{https://www.quark.sk/10-x-naj-za-rok-2015/}{“10 × NAJ za rok 2015"}, \emph{Quark}\\
\years{2017}  The pop-science article about newly described species in 2016, \href{https://www.quark.sk/10-x-naj-za-rok-2016/}{“10 × NAJ za rok 2016"}, \emph{Quark}\\
\years{2018}  The pop-science article about newly described species in 2017, \href{https://www.quark.sk/top-10-za-rok-2017/}{“Top 10 za rok 2017"}, \emph{Quark}\\
\years{2019}  The pop-science article about the extinct cockroach family Alienopteridae, \href{https://www.quark.sk/votrelci-z-jantaru/}{“Votrelci z jantáru"}, \emph{Quark}\\
\years{2021}  The website dedicated to paleontology and paleoentomology (under development) \href{https://paleoentomology.rocks/}{paleoentomology.rocks}

%----------------------------------------------------------------------------------------
%	TEACHING
%----------------------------------------------------------------------------------------

\section*{Teaching}
\begin{itemize}
\itemsep -0.5em
\item 1 hour lesson on the Evolution of Insects as part of the Advanced Systematic Paleontology class  
 \item1/2 hour lesson on the Amber Taphonomy as part of the class Taphonomy and Lagerstatten 
\end{itemize}
-both given at the Institue of Geology and Palaeontology, Charles University
%------------------------------------------------

 %----------------------------------------------------------------------------------------
%	WORK EXPERIENCE
%----------------------------------------------------------------------------------------

\section*{Work experience}
\years{12/2015} Internship at the Institute of Geology and Palaeontology, Chinese Academy of Sciences, Nanjing \\
\years{8/2016} Internship at the Institute of Geology and Palaeontology, Chinese Academy of Sciences, Nanjing \\
\years{10/2017} Internship at the Institute of Geology and Palaeontology, Chinese Academy of Sciences, Nanjing \\
\years{2018-2019} Technical support representative for spanish clients at a stock exchange company \\
\years{2019-now} Creating maps for a \href{http://www.zaniklekrajiny.cz}{project} dedicated to historical demography of Czech republic

%----------------------------------------------------------------------------------------
%	SKILLS AND ABILITIES
%----------------------------------------------------------------------------------------

\section*{Skills and abilities}

\begin{multicols}{2}
\begin{itemize}
\itemsep -0.5em
\item programming with Python, HTML, CSS, JavaScript, \LaTeX 
\item creating maps with ArcGis 
\item drawing/digital drawing and painting (Adobe Fresco, Vectornator, Gimp) 
	
\end{itemize}
\end{multicols}
	\begin{barchart}{5.5}
	\baritem{Slovak}{100}
	\baritem{English}{80}
	\baritem{Spanish}{50}

\end{barchart}

\vfill % Whitespace before final footer

%----------------------------------------------------------------------------------------
%	FINAL FOOTER
%----------------------------------------------------------------------------------------

% Any final footer text such as a URL to the latest version of this CV, last updated date, compiled in XeTeX, etc
\begin{center}
	\scriptsize
	Last updated: \today~~\raisebox{-0.5pt}{\textbullet}~~ \href{https://www.LaTeXTemplates.com}{https://www.LaTeXTemplates.com}
\end{center}

%----------------------------------------------------------------------------------------

\end{document}
